
\documentclass[a4paper,12pt]{article}
\usepackage[utf8]{inputenc}
\usepackage{graphicx}
\usepackage{amsmath}
\usepackage{amssymb}
\usepackage{hyperref}
\usepackage[utf8]{inputenc}
\usepackage[T1]{fontenc}
\usepackage{textcomp}
\usepackage{listings}
\usepackage{color}
\usepackage{fancyvrb}
\usepackage{geometry}
\usepackage{longtable}
\usepackage{titlesec}
\geometry{margin=2.5cm}
\definecolor{lightgray}{gray}{0.95}
\titleformat{\section}[block]{\large\bfseries}{\thesection}{1em}{}
\titleformat{\subsection}[block]{\normalsize\bfseries}{\thesubsection}{1em}{}
\lstset{
  backgroundcolor=\color{lightgray},
  basicstyle=\ttfamily\small,
  breaklines=true,
  frame=single,
  postbreak=\mbox{\textcolor{red}{$\hookrightarrow$}\space}
}

\title{Otsu-Segmentierungsprojekt: Technische Dokumentation}
\author{}
\date{\today}

\begin{document}

\maketitle

\tableofcontents

\newpage

\section{Einleitung}
Dieses Dokument beschreibt ein Projekt zur Zellkernsegmentierung in Mikroskopiebildern unter Verwendung verschiedener Otsu-Verfahren. Es dokumentiert Quellcode, Methodik und Ergebnisse zur Bewertung.

\newpage


\newpage

\section{Zielsetzung}
Ziel des Projekts ist es, Zellkerne aus Fluoreszenz-Mikroskopieaufnahmen automatisch zu segmentieren. Es werden Otsu-basierte Schwellenwertmethoden getestet und gegen Ground-Truth-Masken evaluiert.


\section{Modulübersicht}
(Der gesamte Abschnitt wird aufgeteilt in Unterabschnitte, z. B. 	{dice\_score.py}, 	{gray\_hist.py} usw.)

\section{Modul: \texttt{dice\_score.py} - Berechnung des Dice-Koeffizienten}

\subsection*{Zweck}
Dieses Skript berechnet den \textbf{Dice-Koeffizienten}, eine gängige Metrik zur Bewertung der Überlappung zwischen zwei binären Segmentierungen (z.\,B.\ Vorhersage vs.\ Ground Truth). Der Dice Score liegt zwischen 0 (keine Überlappung) und 1 (perfekte Übereinstimmung).

\subsection*{Importe}
\begin{verbatim}
import numpy as np
from skimage.io import imread
\end{verbatim}

\begin{itemize}
  \item \texttt{numpy} wird für numerische Operationen auf Arrays verwendet.
  \item \texttt{skimage.io.imread} lädt Bilddateien als NumPy-Arrays.
\end{itemize}

\subsection*{Funktion \texttt{dice\_score(pred, target)}}
\begin{verbatim}
def dice_score(pred: np.ndarray, target: np.ndarray) -> float:
\end{verbatim}

\paragraph{Argumente:}
\begin{itemize}
  \item \texttt{pred}: vorhergesagte binäre Maske (z.\,B.\ von einem Segmentierungsalgorithmus), Typ: \texttt{np.ndarray}, dtype: \texttt{bool}
  \item \texttt{target}: Ground-Truth-Maske (manuell annotiert), ebenfalls ein binäres \texttt{np.ndarray}
\end{itemize}

\paragraph{Ablauf:}
\begin{enumerate}
  \item \textbf{Formprüfung}
\begin{verbatim}
if pred.shape != target.shape:
    raise ValueError("Die Eingabebilder haben unterschiedliche Formen.")
\end{verbatim}

  \item \textbf{Berechnung der Überlappung (Intersection)}
\begin{verbatim}
intersection = np.logical_and(pred, target).sum()
\end{verbatim}

  \item \textbf{Berechnung der Gesamtanzahl positiver Pixel}
\begin{verbatim}
total = pred.sum() + target.sum()
\end{verbatim}

  \item \textbf{Sonderfallbehandlung}
\begin{verbatim}
if total == 0:
    return 1.0
\end{verbatim}

  \item \textbf{Berechnung des Dice Scores}
\begin{verbatim}
return 2 * intersection / total
\end{verbatim}
\[
\text{Dice} = \frac{2 \cdot |A \cap B|}{|A| + |B|}
\]
\end{enumerate}

\subsection*{Testlauf im \texttt{\_\_main\_\_}-Block}
\begin{verbatim}
if __name__ == "__main__":
    pred = imread("data-git/N2DH-GOWT1/img/t01.tif", as_gray=True) > 0
    gt   = imread("data-git/N2DH-GOWT1/gt/man_seg01.tif", as_gray=True) > 0
    print(f"Dice Score: {dice_score(pred, gt):.4f}")
\end{verbatim}

\paragraph{Erklärung:}
\begin{itemize}
  \item \texttt{imread(..., as\_gray=True)} lädt das Bild als Graustufenbild.
  \item \texttt{> 0} binarisiert das Bild (alle Pixel $> 0$ werden zu \texttt{True}, alle anderen zu \texttt{False}).
  \item Danach wird \texttt{dice\_score(pred, gt)} berechnet und mit 4 Dezimalstellen ausgegeben.
\end{itemize}

\paragraph{Beispielausgabe:}
\begin{verbatim}
Dice Score: 0.8234
\end{verbatim}
Dies bedeutet: Es besteht eine Überlappung von etwa 82{,}34\,\% zwischen vorhergesagter und tatsächlicher Segmentierung.

\subsection*{Zusammenfassung}
\begin{itemize}
  \item Diese Funktion ist \textbf{zentral für die quantitative Bewertung} von Segmentierungsergebnissen.
  \item Sie ist robust gegenüber leeren Masken.
  \item Sie kann leicht in größere Pipelines eingebunden werden, um \textbf{viele Segmentierungsmethoden vergleichbar zu machen}.
\end{itemize}

\section{Modul: \texttt{gray\_hist.py} – Grauwert-Histogrammberechnung}

Dieses Modul dient der Berechnung und optionalen Visualisierung von Histogrammen für Graustufenbilder. Es wird unter anderem für die Otsu-Segmentierung verwendet, um die Intensitätsverteilung eines Bildes auszuwerten.

\subsection*{Funktionen}

\subsubsection*{\texttt{compute\_gray\_histogram(image\_source, bins=256, value\_range=(0, 255))}}

\paragraph{Zweck}
Diese Funktion berechnet das Histogramm eines Grauwertbildes.

\paragraph{Eingaben:}
\begin{itemize}
  \item \texttt{image\_source} (\texttt{Path}, \texttt{str} oder \texttt{np.ndarray}): Das Bild kann entweder als Pfad oder direkt als NumPy-Array übergeben werden.
  \item \texttt{bins} (\texttt{int}): Anzahl der Bins im Histogramm. Standard: 256 (für 8-Bit-Bilder sinnvoll).
  \item \texttt{value\_range} (\texttt{Tuple[int, int]}): Wertebereich der Intensitäten. Standardmäßig von 0 bis 255.
\end{itemize}

\paragraph{Ausgabe:}
\begin{itemize}
  \item \texttt{hist}: Array mit den Häufigkeiten der Grauwerte.
  \item \texttt{bin\_edges}: Array mit den Bin-Grenzen.
\end{itemize}

\paragraph{Funktionsweise:}
\begin{enumerate}
  \item Das Bild wird geladen (falls ein Pfad übergeben wurde).
  \item Es wird in ein Grauwertbild konvertiert.
  \item Das Array wird mit \texttt{.ravel()} flach gemacht.
  \item Das Histogramm wird mit \texttt{np.histogram()} berechnet.
\end{enumerate}

\subsubsection*{\texttt{plot\_gray\_histogram(hist, bin\_edges)}}

\paragraph{Zweck}
Visualisiert ein Histogramm mit \texttt{matplotlib}.

\paragraph{Eingaben:}
\begin{itemize}
  \item \texttt{hist}: Die berechneten Häufigkeiten (z.\,B.\ aus \texttt{compute\_gray\_histogram}).
  \item \texttt{bin\_edges}: Die zugehörigen Intensitätsgrenzen.
\end{itemize}

\paragraph{Funktionsweise:}
\begin{itemize}
  \item Erstellt ein Balkendiagramm mit \texttt{plt.bar}.
  \item Setzt Achsenbeschriftungen: „Grauwert“ und „Häufigkeit“.
  \item Zeigt das Diagramm mit \texttt{plt.show()} an.
\end{itemize}

\subsection*{Beispielanwendung}

\begin{verbatim}
from gray_hist import compute_gray_histogram, plot_gray_histogram
hist, bin_edges = compute_gray_histogram("path/to/image.png")
plot_gray_histogram(hist, bin_edges)
\end{verbatim}

\subsection*{Hinweise}
\begin{itemize}
  \item Diese Funktionen sind wichtig für Schwellenwertverfahren wie das Otsu-Verfahren.
  \item Sie erlauben eine Analyse der Bildhelligkeit und helfen bei der automatischen Segmentierung.
\end{itemize}

\section{Modul: \texttt{load\_image\_pair.py}}

Dieses Skript enthält eine zentrale Hilfsfunktion zum Laden von Bild-Ground-Truth-Paaren aus einem gegebenen Dateipfad. Es wird verwendet, um sowohl das Eingabebild (z.\,B.\ ein Graustufen-Mikroskopiebild) als auch die zugehörige Ground-Truth-Segmentierungsmaske korrekt zu laden und als \texttt{NumPy}-Arrays zurückzugeben.

\subsection*{Ziel}

Die Funktion stellt sicher, dass:
\begin{itemize}
  \item Bilder korrekt als Grauwertbilder gelesen werden,
  \item die Ground-Truth-Maske korrekt binarisiert wird (also in ein boolesches Format überführt wird),
  \item die Rückgabe für weitere Segmentierungs- und Evaluierungsschritte geeignet ist.
\end{itemize}

\subsection*{Code-Übersicht}

\begin{verbatim}
import numpy as np
from skimage.io import imread
from typing import Tuple, Union
from pathlib import Path
\end{verbatim}

\subsubsection*{Funktion: \texttt{load\_image\_and\_gt(...)}}

\begin{verbatim}
def load_image_and_gt(
    image_path: Union[str, Path],
    gt_path: Union[str, Path],
    threshold: float = 0.0
) -> Tuple[np.ndarray, np.ndarray]:
\end{verbatim}

\paragraph{Parameter:}
\begin{itemize}
  \item \texttt{image\_path}: Pfad zum Bild (String oder Pathlib-Objekt)
  \item \texttt{gt\_path}: Pfad zur Ground-Truth-Maske
  \item \texttt{threshold}: Schwellenwert, um die GT-Maske in eine binäre Maske umzuwandeln (Standard: 0.0)
\end{itemize}

\paragraph{Rückgabe:}
Ein Tupel bestehend aus:
\begin{itemize}
  \item \texttt{image}: Graustufenbild als 2D \texttt{np.ndarray} mit Werten im Bereich [0,1]
  \item \texttt{gt\_mask}: Binäre Ground-Truth-Maske (\texttt{dtype=bool})
\end{itemize}

\subsection*{Funktionsweise im Detail}

\begin{verbatim}
image = imread(str(image_path), as_gray=True)
\end{verbatim}

Das Bild wird mithilfe von \texttt{skimage.io.imread} als Graustufenbild geladen (automatisch normalisiert auf Bereich [0,1]).

\begin{verbatim}
gt_mask = imread(str(gt_path), as_gray=True) > threshold
\end{verbatim}

Die Ground-Truth-Maske wird ebenfalls als Grauwertbild geladen. Durch den Vergleich \texttt{> threshold} wird das Bild in eine binäre Maske umgewandelt, z.\,B.\ alles was größer als 0 ist, wird als „True“ interpretiert.

\subsection*{Beispiel}

\begin{verbatim}
image, gt = load_image_and_gt("data/N2DH-GOWT1/img/t01.tif", 
"data/N2DH-GOWT1/gt/man_seg01.tif")
\end{verbatim}

Lädt das Bild \texttt{t01.tif} und die zugehörige Ground-Truth-Maske \texttt{man\_seg01.tif} aus dem Dataset \texttt{N2DH-GOWT1}.

\subsection*{Hinweise}

\begin{itemize}
  \item Der Schwellenwert kann angepasst werden, falls GT-Bilder in Grauwerten vorliegen.
  \item Diese Funktion stellt sicher, dass sowohl Bild als auch Maske kompatibel weiterverarbeitet werden können.
\end{itemize}

\section{Modul: \texttt{otsu\_global.py}}

Dieses Modul enthält die eigene Implementierung des globalen Otsu-Verfahrens zur Schwellenwertbestimmung und Segmentierung von Grauwertbildern.


\subsection*{Funktionen}

\subsubsection*{\texttt{otsu\_threshold(p: np.ndarray) -> int}}

Berechnet den optimalen Schwellenwert $t$ anhand einer Wahrscheinlichkeitsverteilung $p$ (normalisiertes Histogramm).

\paragraph{Ablauf:}
\begin{enumerate}
  \item $P = \texttt{np.cumsum(p)}$ \quad (kumulative Summe der Wahrscheinlichkeiten)
  \item $\texttt{bins} = \texttt{np.arange(len(p))}$ \quad (Grauwertachsen)
  \item $\mu = \texttt{np.cumsum(bins * p)}$ \quad (kumulative Mittelwerte)
  \item $\mu_T = \mu[-1]$ \quad (Gesamtmittelwert)
  \item $\sigma_b^2 = \frac{(\mu_T \cdot P - \mu)^2}{P \cdot (1 - P) + 10^{-12}}$ \quad (Zwischenklassenvarianz)
  \item $\texttt{argmax}(\sigma_b^2)$ \quad liefert den optimalen Schwellenwert
\end{enumerate}

\subsubsection*{\texttt{binarize(arr: np.ndarray, t: int) -> np.ndarray}}

Binarisiert ein Grauwertbild: Alle Pixel größer als $t$ werden als 1 (Objekt) gesetzt.

\textbf{Rückgabe:} Binärbild (0 = Hintergrund, 1 = Objekt)

\subsubsection*{\texttt{apply\_global\_otsu(image: np.ndarray) -> np.ndarray}}

Vollständige Pipeline:
\begin{enumerate}
  \item Berechnung des Histogramms mit \texttt{compute\_gray\_histogram(...)}
  \item Normalisierung zu Wahrscheinlichkeiten
  \item Bestimmung des Schwellenwertes mit \texttt{otsu\_threshold(...)}
  \item Binarisierung mit \texttt{binarize(...)}
\end{enumerate}

\subsection*{Beispielverwendung}

\begin{verbatim}
from otsu_global import apply_global_otsu
from skimage.io import imread
image = imread("path/to/image.tif", as_gray=True)
binary = apply_global_otsu(image)
\end{verbatim}

\subsection*{Abhängigkeiten}
\begin{itemize}
  \item \texttt{gray\_hist.py} (Histogrammberechnung)
  \item \texttt{numpy}
\end{itemize}

\section{Modul: \texttt{otsu\_local.py}}

Dieses Modul implementiert eine lokale Version des Otsu-Schwellenwertverfahrens. Statt eines globalen Schwellenwertes wird für jedes Pixel ein individueller Schwellenwert anhand seines lokalen Umfelds berechnet.

\subsection*{Funktion: \texttt{local\_otsu(image, radius=3)}}

\paragraph{Parameter:}
\begin{itemize}
  \item \texttt{image} (\texttt{np.ndarray}): Eingabebild im Wertebereich $[0, 1]$
  \item \texttt{radius} (\texttt{int}): Umgebungsradius (Standard: 3), resultiert in einem Block der Größe $(2r + 1)^2$
\end{itemize}

\paragraph{Rückgabewerte:}
\begin{itemize}
  \item \texttt{t\_map} (\texttt{np.ndarray}): Karte lokaler Schwellenwerte
  \item \texttt{mask} (\texttt{np.ndarray}): Binäre Maske (\texttt{True} = Objekt)
\end{itemize}

\subsection*{Funktionsweise}

\begin{enumerate}
  \item Vorverarbeitung des Bildes zu 8-Bit mit \texttt{skimage.img\_as\_ubyte}
  \item Padding des Bildes mit \texttt{np.pad}, um Randverarbeitung zu ermöglichen
  \item Für jedes Pixel:
  \begin{itemize}
    \item Extraktion des lokalen Blocks
    \item Berechnung des Grauwert-Histogramms
    \item Anwendung des Otsu-Verfahrens (mit \texttt{otsu\_global.otsu\_threshold})
    \item Vergleich des Pixelwerts mit lokalem Schwellenwert
  \end{itemize}
  \item Rückgabe von Schwellenwertkarte und Segmentierungsmaske
\end{enumerate}

\subsection*{Abhängigkeiten}
\begin{itemize}
  \item \texttt{numpy}
  \item \texttt{skimage.img\_as\_ubyte}
  \item \texttt{src.gray\_hist.compute\_gray\_histogram}
  \item \texttt{src.otsu\_global.otsu\_threshold}
\end{itemize}

\textbf{Vorteil:} Diese Methode ist besonders robust bei inhomogener Ausleuchtung oder variierenden Kontrasten, da sie kontextabhängig segmentiert.


\section*{ Modulbeschreibung: \texttt{process\_image.py}, \texttt{process\_image\_ein.py}, \texttt{process\_image\_all.py}}

Dieses Modul-Set bündelt alle Schritte zur Anwendung und Auswertung verschiedener Bildsegmentierungsmethoden auf einzelne oder mehrere Mikroskopie-Bilder. Im Fokus stehen Otsu-basierte Verfahren sowie Referenzimplementierungen aus \texttt{skimage}.

\subsection*{ \texttt{process\_image.py}}

\subsubsection*{Funktion \texttt{process\_all\_methods(image)}}

Diese Funktion nimmt ein einzelnes Grauwertbild entgegen und wendet folgende Segmentierungsmethoden darauf an:

\begin{itemize}
  \item \textbf{Otsu Global (custom)}: Eigene Implementierung basierend auf Histogramm-Analyse.
  \item \textbf{Otsu Local (custom)}: Berechnet für jedes Pixel einen lokalen Schwellenwert basierend auf Histogrammen im Umkreis.
  \item \textbf{Otsu Global (skimage)}: Standard-Implementierung aus \texttt{skimage.filters.threshold\_otsu}.
  \item \textbf{Otsu Local (skimage)}: Lokale Schwellenwertmethode (\texttt{threshold\_local}) mit anpassbarem Fenster.
  \item \textbf{Multi-Otsu (skimage)}: Erweiterung für mehr als zwei Klassen – nützlich zur Trennung von Zellkern, Zytoplasma und Hintergrund.
\end{itemize}

\textbf{Rückgabewert:} Ein Dictionary \texttt{\{Methodenname: Binärmaske (np.ndarray)\}}

\paragraph{Helferfunktionen:}
\begin{itemize}
  \item \texttt{apply\_skimage\_global(...)} – verwendet globalen Otsu aus \texttt{skimage}
  \item \texttt{apply\_skimage\_local(...)} – verwendet lokalen Schwellenwert aus \texttt{skimage}
  \item \texttt{apply\_skimage\_multiotsu(...)} – verwendet \texttt{threshold\_multiotsu} für mehrklassige Segmentierung
\end{itemize}

\subsection*{ \texttt{process\_image\_ein.py}}

\paragraph{Ziel:}
\begin{itemize}
  \item Testweise Anwendung der Methoden auf genau ein Bild.
  \item Nutzt \texttt{load\_image\_and\_gt(...)} um Bild und Ground Truth zu laden.
  \item Führt \texttt{process\_all\_methods(image)} aus.
  \item Gibt \textbf{Form} und \textbf{Anzahl positiver Pixel} jeder Methode in der Konsole aus.
\end{itemize}

\paragraph{Beispielausgabe:}
\begin{verbatim}
Otsu Global (custom): (512, 512), Positiv: 12034
Otsu Local (custom):  (512, 512), Positiv: 13400
...
\end{verbatim}

\subsection*{ \texttt{process\_image\_all.py}}

\paragraph{Ziel:}
\begin{itemize}
  \item Läuft automatisiert über alle Datensätze im Verzeichnis \texttt{data/}
  \item Erkennt automatisch, ob ein Bild zur \textbf{NIH3T3}-Serie gehört (andere Benennung)
  \item Lädt jedes Bild mit zugehöriger Ground Truth
  \item Führt \texttt{process\_all\_methods(image)} aus
  \item Gibt pro Bild und Methode die Dimension und Anzahl der Segmentierungspixel in der Konsole aus
\end{itemize}

\paragraph{Hinweise zur Funktion:}
\begin{itemize}
  \item Unterstützt \texttt{.tif} und \texttt{.png}
  \item Erkennt fehlende Ground-Truth-Dateien automatisch und überspringt diese
  \item Gut geeignet für erste visuelle Kontrolle der Pipeline auf kompletten Datensätzen
\end{itemize}

\subsection*{ Zusammenfassung}

Diese drei Dateien bilden gemeinsam das Rückgrat der Bildverarbeitungspipeline:

\begin{center}
\begin{tabular}{|l|l|}
\hline
\textbf{Datei} & \textbf{Aufgabe} \\
\hline
\texttt{process\_image.py} & Definiert alle Segmentierungsmethoden \\
\texttt{process\_image\_ein.py} & Einzelbild-Analyse (Debugging, Test) \\
\texttt{process\_image\_all.py} & Batch-Anwendung auf komplette Datensätze \\
\hline
\end{tabular}
\end{center}


\section*{ \texttt{evaluate\_segmentation.py} – Auswertung der Segmentierung mittels Dice Score}

Dieses Modul bewertet die Qualität verschiedener Segmentierungsmethoden, indem es den sogenannten \textbf{Dice Score} verwendet. Der Dice-Koeffizient ist ein gängiges Maß zur Überlappung binärer Masken in der Bildverarbeitung.

\subsection*{ Funktionsweise}

\subsubsection*{Funktion: \texttt{evaluate\_segmentations(gt\_mask, predictions)}}

Diese Funktion berechnet den Dice Score für jede Segmentierungsmethode im Vergleich zur Ground-Truth-Maske.

\paragraph{Argumente:}
\begin{itemize}
  \item \texttt{gt\_mask} (\texttt{np.ndarray}): Die Ground-Truth-Maske (boolesches Array)
  \item \texttt{predictions} (\texttt{Dict[str, np.ndarray]}): Ein Dictionary mit Namen der Methoden und den binären Segmentierungsmasken
\end{itemize}

\paragraph{Rückgabe:}
\begin{itemize}
  \item \texttt{pd.DataFrame}: Eine sortierte Tabelle mit den Dice Scores aller Methoden
\end{itemize}

\paragraph{Rechenprinzip:} Für jede Methode wird die Vorhersagemaske in ein boolesches Format umgewandelt. Dann wird mithilfe der Funktion \texttt{dice\_score} die Übereinstimmung zur Ground Truth berechnet.

\subsection*{ Beispiel: Anwendung auf ein einzelnes Bild}

\begin{verbatim}
from src.load_image_pair import load_image_and_gt
from process_image import process_all_methods
from evaluate_segmentation import evaluate_segmentations

image, gt_mask = load_image_and_gt("data/N2DH-GOWT1/img/t01.tif",
                                   "data/N2DH-GOWT1/gt/man_seg01.tif")

results = process_all_methods(image)
df_scores = evaluate_segmentations(gt_mask, results)
print(df_scores)
\end{verbatim}

\subsection*{ Anwendung auf ganze Datensätze}

Ein weiterer Abschnitt des Codes iteriert durch alle Bilder eines Datensatzes. Dabei wird für jedes Bild die Segmentierung berechnet, mit der Ground Truth verglichen und die Dice Scores ausgegeben.

\paragraph{Ablauf:}
\begin{enumerate}
  \item Iteration über alle Ordner in \texttt{data/}
  \item Laden aller Bilder aus \texttt{img/}
  \item Zuordnung der passenden Ground-Truth-Datei aus \texttt{gt/}
  \item Aufruf von:
    \begin{itemize}
      \item \texttt{load\_image\_and\_gt}
      \item \texttt{process\_all\_methods}
      \item \texttt{evaluate\_segmentations}
    \end{itemize}
  \item Ausgabe der Dice Scores pro Bild und Methode
\end{enumerate}

Dieser automatisierte Ablauf dient zur vergleichenden Evaluation über ganze Datensätze hinweg.

\subsection*{ Fehlerbehandlung}
\begin{itemize}
  \item Bilder ohne passende Ground Truth werden übersprungen
  \item Fehler beim Laden oder Verarbeiten eines Bildes werden abgefangen und ausgegeben
\end{itemize}

\subsection*{ Nutzen}

Dieses Skript ist essenziell für die \textbf{quantitative Bewertung} der Segmentierungsmethoden. Es zeigt zuverlässig, welche Methode in welchem Datensatz die besten Ergebnisse liefert.

\section*{ Dokumentation: \texttt{run\_batch\_evaluation.py}}

Dieses Skript automatisiert die Auswertung von Segmentierungsmethoden durch Vergleich der segmentierten Masken mit Ground-Truth-Daten. Es verwendet den Dice-Koeffizienten zur Bewertung der Genauigkeit.

\section*{ Funktionen in \texttt{run\_batch\_evaluation.py}}

\subsection*{Funktion: \texttt{run\_batch\_evaluation(img\_dir, gt\_dir, dataset=None)}}

\begin{itemize}
  \item \textbf{Zweck:} Führt Segmentierung und Bewertung für alle Bilder eines Datensatzes durch.
  \item \textbf{Argumente:}
    \begin{itemize}
      \item \texttt{img\_dir}: Pfad zum Ordner mit Input-Bildern.
      \item \texttt{gt\_dir}: Pfad zum Ordner mit Ground-Truth-Masken.
      \item \texttt{dataset} (optional): Name des Datensatzes (für Logging/Export).
    \end{itemize}
  \item \textbf{Rückgabe:} \texttt{pandas.DataFrame} mit Spalten: \texttt{Bild}, \texttt{Methode}, \texttt{Dice Score}, \texttt{Datensatz} (optional).
\end{itemize}

\subsubsection*{Ablauf:}

\begin{enumerate}
  \item \textbf{Dateinamen sammeln:} Bilddateien (\texttt{.tif}, \texttt{.png}) und GT-Dateien werden gesammelt.
  \item \textbf{Matching:} Die passende GT-Datei wird anhand des Bildnamens abgeleitet.
  \item \textbf{Laden \& Segmentieren:} Für jedes Paar:
    \begin{itemize}
      \item Bild und GT laden (\texttt{load\_image\_and\_gt})
      \item Alle Methoden auf das Bild anwenden (\texttt{process\_all\_methods})
      \item Dice Score für jede Methode berechnen (\texttt{evaluate\_segmentations})
    \end{itemize}
  \item \textbf{Daten sammeln:} Ergebnisse werden gesammelt und in einem DataFrame zurückgegeben.
\end{enumerate}

\subsection*{ Anwendungsskript: Einzelordner}

\begin{verbatim}
from run_batch_evaluation import run_batch_evaluation
import os

if __name__ == "__main__":
    img_dir = "data/N2DH-GOWT1/img"
    gt_dir  = "data/N2DH-GOWT1/gt"

    df = run_batch_evaluation(img_dir, gt_dir)
    os.makedirs("results", exist_ok=True)
    df.to_csv("results/dice_scores.csv", index=False)
    print(df)
\end{verbatim}

\subsection*{ Anwendungsskript: Alle Datensätze}

\begin{verbatim}
from run_batch_evaluation import run_batch_evaluation
import os
import pandas as pd

if __name__ == "__main__":
    base_data_dir = "data"
    all_dfs = []

    for dataset_name in os.listdir(base_data_dir):
        dataset_path = os.path.join(base_data_dir, dataset_name)
        img_dir = os.path.join(dataset_path, "img")
        gt_dir  = os.path.join(dataset_path, "gt")

        if not (os.path.isdir(img_dir) and os.path.isdir(gt_dir)):
            print(f"  Überspringe {dataset_name}, 'img/' oder 'gt/' fehlt.")
            continue

        print(f" Verarbeite Datensatz: {dataset_name}")
        try:
            df = run_batch_evaluation(img_dir, gt_dir, dataset=dataset_name)
            all_dfs.append(df)
        except Exception as e:
            print(f"    Fehler bei {dataset_name}: {e}")

    if all_dfs:
        df_all = pd.concat(all_dfs, ignore_index=True)
        os.makedirs("results", exist_ok=True)
        df_all.to_csv("results/dice_scores.csv", index=False)
        print(f" Ergebnisse gespeichert: results/dice_scores.csv")
    else:
        print(" Keine Ergebnisse gesammelt.")
\end{verbatim}

\subsection*{ Hinweise}

\begin{itemize}
  \item Es wird das Modul \texttt{tqdm} für Fortschrittsanzeigen verwendet.
  \item Ergebnisse werden in \texttt{results/dice\_scores.csv} gespeichert.
  \item Dieses Skript eignet sich sowohl für Einzel- als auch Mehrfachdatensätze.
\end{itemize}

\section*{ Visualisierung \& Analyse}

\subsection*{ Visualisierung von Segmentierungen (\texttt{visualize\_segmentations.py})}

Dieses Modul dient der Darstellung und Abspeicherung von Segmentierungsergebnissen aus dem Projekt zur Zellkern-Segmentierung. Es ermöglicht die vergleichende Visualisierung zwischen dem Originalbild, der Ground Truth und den Resultaten verschiedener Segmentierungsmethoden.

\subsection*{ Funktion: \texttt{visualize\_segmentations(...)}}

\begin{verbatim}
def visualize_segmentations(
    image: np.ndarray,
    gt_mask: np.ndarray,
    predictions: Dict[str, np.ndarray],
    max_cols: int = 3,
    save_path: str = None
)
\end{verbatim}

\textbf{Parameter:}
\begin{itemize}
  \item \texttt{image}: Das Originalbild in Graustufen (float, skaliert auf [0, 1])
  \item \texttt{gt\_mask}: Die Ground-Truth-Maske als binäres Bild (boolesches Array)
  \item \texttt{predictions}: Ein Dictionary mit Methodenname $\rightarrow$ Segmentierungsmaske
  \item \texttt{max\_cols}: Maximale Anzahl an Spalten in der Darstellungsübersicht (Standard: 3)
  \item \texttt{save\_path}: Optionaler Dateipfad zur Abspeicherung der Visualisierung
\end{itemize}

\textbf{Ablauf:}
\begin{enumerate}
  \item Erstellung einer kombinierten Liste mit Titeln und zugehörigen Bildern: \texttt{"Original"}, \texttt{"Ground Truth"} und allen Einträgen aus \texttt{predictions}.
  \item Berechnung der benötigten Zeilen und Spalten.
  \item Für jede Bild-Maske-Kombination wird ein \texttt{subplot} erzeugt:
  \begin{itemize}
    \item Bildanzeige, Titel setzen, Achsen entfernen.
  \end{itemize}
  \item Nicht benutzte Subplots werden deaktiviert.
  \item Optional: Abspeichern des Gesamtergebnisses als PNG-Datei.
\end{enumerate}

\subsection*{ Beispielverwendung (Einzelbild)}

\begin{verbatim}
from src.load_image_pair import load_image_and_gt
from process_image import process_all_methods
from visualize_segmentation import visualize_segmentations

image, gt_mask = load_image_and_gt(
    "data/N2DH-GOWT1/img/t01.tif",
    "data/N2DH-GOWT1/gt/man_seg01.tif"
)

results = process_all_methods(image)

visualize_segmentations(image, gt_mask, results)
\end{verbatim}

\subsection*{ Batch-Visualisierung aller Bilder}

Ein erweiterter Codeabschnitt visualisiert und speichert automatisch die Segmentierungen aller Bilder eines Datensatzes.

\textbf{Wichtige Punkte:}
\begin{itemize}
  \item Alle \texttt{.tif} und \texttt{.png} Bilder aus dem Verzeichnis \texttt{img/} werden geladen.
  \item Ground-Truth-Dateien werden automatisch anhand des Bildnamens zugeordnet.
  \item Für jedes Bild und jede Methode wird eine Visualisierung mit Originalbild, GT und Segmentierung erzeugt.
  \item Die Ergebnisse werden in \texttt{output\_visuals/} gespeichert.
\end{itemize}

\textbf{Codeausschnitt:}
\begin{verbatim}
for method_name, mask in predictions.items():
    fig, axes = plt.subplots(1, 3, figsize=(12, 4))
    axes[0].imshow(image, cmap="gray")
    axes[0].set_title("Originalbild")
    axes[1].imshow(gt_mask, cmap="gray")
    axes[1].set_title("Ground Truth")
    axes[2].imshow(mask, cmap="gray")
    axes[2].set_title(f"Segmentierung: {method_name}")
    ...
    plt.savefig(out_path)
\end{verbatim}

\subsection*{ Ergebnis}

Die gespeicherten Bilder befinden sich im Verzeichnis \texttt{output\_visuals/\{Datensatz\}/\{Methode\}/}. Jede PNG-Datei enthält eine vergleichende Darstellung zwischen Originalbild, Ground Truth und segmentiertem Bild.

\subsection*{ Fehlerbehandlung}

Falls Ground-Truth-Dateien fehlen oder ein Bild fehlerhaft ist, wird dies mit einer Warnmeldung im Terminal angezeigt. Die Verarbeitung wird dennoch fortgesetzt.


\section{Visualisierung und Analyse der Segmentierungsmethoden}

\subsection{Visualisierung von Segmentierungen (\texttt{visualize\_segmentations.py})}

Dieses Modul dient der Darstellung und Abspeicherung von Segmentierungsergebnissen aus dem Projekt zur Zellkern-Segmentierung. Es ermöglicht die vergleichende Visualisierung zwischen dem Originalbild, der Ground Truth und den Resultaten verschiedener Segmentierungsmethoden.

\subsubsection*{Funktion \texttt{visualize\_segmentations(...)}} 

\begin{verbatim}
def visualize_segmentations(
    image: np.ndarray,
    gt_mask: np.ndarray,
    predictions: Dict[str, np.ndarray],
    max_cols: int = 3,
    save_path: str = None
)
\end{verbatim}

\paragraph{Parameter:}
\begin{itemize}
    \item \texttt{image}: Originalbild in Graustufen, skaliert auf [0, 1].
    \item \texttt{gt\_mask}: Ground-Truth-Maske als binäres Array.
    \item \texttt{predictions}: Dictionary mit Methodenname \textrightarrow{} Segmentierungsmaske.
    \item \texttt{max\_cols}: Maximale Spaltenanzahl für die Darstellung.
    \item \texttt{save\_path}: Optionaler Pfad zur Speicherung der Visualisierung.
\end{itemize}

\paragraph{Ablauf:}
\begin{enumerate}
    \item Erstellung einer kombinierten Liste aus Originalbild, Ground Truth und den Segmentierungen.
    \item Berechnung der nötigen Zeilen- und Spaltenanzahl.
    \item Darstellung mittels Subplots und Titelbeschriftung.
    \item Speicherung (optional) im PNG-Format.
\end{enumerate}

\paragraph{Beispielverwendung (Einzelbild):}

\begin{verbatim}
from src.load_image_pair import load_image_and_gt
from process_image import process_all_methods
from visualize_segmentation import visualize_segmentations

image, gt_mask = load_image_and_gt(
    "data/N2DH-GOWT1/img/t01.tif",
    "data/N2DH-GOWT1/gt/man_seg01.tif"
)

results = process_all_methods(image)
visualize_segmentations(image, gt_mask, results)
\end{verbatim}

\subsubsection*{Batch-Visualisierung}
\begin{itemize}
    \item Lädt alle .tif/.png Bilder in \texttt{img/}.
    \item Erzeugt Visualisierungen für jede Methode.
    \item Ergebnisse werden in \texttt{output\_visuals/} gespeichert.
\end{itemize}

\paragraph{Ergebnis:} PNG-Dateien mit direktem Vergleich zwischen Original, Ground Truth und Segmentierung.

\paragraph{Fehlerbehandlung:} Fehlende GT-Dateien oder fehlerhafte Bilder werden erkannt und übersprungen.

\vspace{1em}
\subsection{Vergleich der Otsu-Methoden (Scatterplots)}

\subsubsection*{Vergleich Otsu Local -- \texttt{plot\_otsu\_local\_comparison.py}}

\begin{itemize}
    \item Lädt \texttt{results/dice\_scores.csv}.
    \item Filtert auf ``Otsu Local (custom)'' und ``Otsu Local (skimage)''.
    \item Erstellt Scatterplot: jeder Punkt ein Bild, Achsen = Dice Score pro Methode.
    \item Speichert Plot als \texttt{results/otsu\_local\_scatterplot.png}.
\end{itemize}

\subsubsection*{Vergleich Otsu Global -- \texttt{plot\_otsu\_global\_comparison.py}}

\begin{itemize}
    \item Gleiches Vorgehen wie oben, aber für ``Otsu Global (custom)'' vs. ``Otsu Global (skimage)''.
    \item Plot zeigt relative Genauigkeit beider globaler Verfahren.
\end{itemize}

\vspace{1em}
\subsection{Auswertung aller Methoden: \texttt{plot\_all\_methods.py}}

\begin{itemize}
    \item Erzeugt zwei zentrale Visualisierungen:
    \begin{itemize}
        \item \textbf{Boxplot:} Dice Score je Methode – Verteilung, Ausreißer, Median.
        \item \textbf{Heatmap:} Dice Score für jede Kombination aus Bild und Methode.
    \end{itemize}
    \item Beide Plots werden im Ordner \texttt{results/} gespeichert.
\end{itemize}

\paragraph{Hinweis:} Stelle sicher, dass die Datei \texttt{results/dice\_scores.csv} vorliegt. Sie wird z.B. durch \texttt{run\_batch\_evaluation.py} erzeugt.

\vspace{1em}
\subsection{Zusammenfassung}

\begin{itemize}
    \item Scatterplots ermöglichen direkte Methode-zu-Methode-Vergleiche.
    \item Boxplots geben Überblick über die Stabilität und Verteilung der Genauigkeit.
    \item Heatmaps identifizieren Methoden, die bei bestimmten Bildern besonders gut/schlecht abschneiden.
\end{itemize}


\end{document}

